\documentclass[aspectratio=169]{beamer}

\mode<presentation> {
\usetheme{Hannover}
  \setbeamertemplate{authors}{}
  % Remove navigation symbols.
  \setbeamertemplate{navigation symbols}{}
}
% Dark theme. Comment if there's lots of ambient light.
\usecolortheme{owl}

\usepackage{color}
\usepackage[utf8]{inputenc}
\usepackage{listings}
\usepackage{graphicx}
\usepackage[font=tiny,skip=0pt]{caption}
\usepackage{verbatim}

\graphicspath{{img/}}

% Use symbols instead of numerals for footnotes.
\renewcommand{\thefootnote}{\fnsymbol{footnote}}
% Reset footnote counter every section.
\makeatletter
\@addtoreset{footnote}{subsection}
\makeatother

\title[Fast-sync in Cuprate]{Fast-sync in Cuprate}
\subtitle{Memory-safe Monero node implementation gets closer to feature parity.}
\author{jomuel \and dllud \thanks{\footnotesize with great support from boog900}}
\date{}
\logo{\includegraphics[scale=0.25]{cuprate-logo}}
%\titlegraphic{\includegraphics[width=16em]{cuprate-wordmark}}

\begin{document}

\begin{frame}
  \maketitle
  Developed for MoneroKon's Hackathon Anti-fragility Award.
\end{frame}

\section{Cuprate}
\begin{frame}[fragile]
  \frametitle{Cuprate}
  \begin{columns}
    \begin{column}{0.8\textwidth}
      Monero full node written from scratch in Rust.
    \end{column}
    \begin{column}{0.2\textwidth}
      \includegraphics[width=\textwidth]{cuprate-logo}
    \end{column}
  \end{columns}
  \vspace{2\baselineskip}
  Brings in client diversity to Monero:
  \begin{itemize}
    \item independent validation of Monero consensus rules
    \item protects network from implementation bugs in monerod\\
      (remember block 202612)
  \end{itemize}
  \vspace{2\baselineskip}
  Written in a memory-safe language: avoids (most) memory errors and related security vulnerabilities.
\end{frame}

\section{Fast-sync}
\subsection{Overview}
\begin{frame}[fragile]
  \frametitle{Fast-sync Overview}
  Block sync acceleration technique. The current default on monerod.\\
  Skips the (expensive) RandomX PoW verification by trusting known block hashes.\\
  Not yet implement in Cuprate.\\
  \vspace{2\baselineskip}
  In a VPS with: 3 vCore, 4GB RAM, 200 Mbit/s network, SSD\footnote{https://gist.github.com/DaWe35/aaa0d1a99be4a6fb0977fb7df7ddb702}
  \begin{itemize}
    \item "Slow" sync: 15 days.
    \item Fast sync: 4 days.
  \end{itemize}
  \vspace{2\baselineskip}
  The catch: you are trusting monerod binary on the old blocks PoW validity.
\end{frame}

\subsection{Hashes of Hashes}
\begin{frame}[fragile]
  \frametitle{Hashes of Hashes}
  Storing each block hash in monerod would bloat the binary too much.\\
  \vspace{1\baselineskip}
  Instead monerod hashes groups of block hashes and stores that.\\
  \vspace{1\baselineskip}
  Current default is 512 hashes per group, which creates a binary blob with 379 KiB.\\
\end{frame}

\subsection{Loop}
\begin{frame}[fragile]
  \frametitle{Loop}
  \begin{enumerate}
    \item Peer sends block hashes.
    \item Hashes get grouped, hashed together and checked for validity.
    \item If valid, the full blocks are requested from peers.
    \item Each individual block hash gets checked.
    \item If it matches, block is inserted into DB with PoW hash as zeros.
  \end{enumerate}
\end{frame}

\section{Implementation}
\subsection{Overview}
\begin{frame}[fragile]
  \frametitle{Implementation Overview}
  Based on detailed instructions by boog900 (Cuprate's main dev).\\
  \url{https://github.com/Cuprate/cuprate/issues/153}\\
  \vspace{2\baselineskip}
  Draft pull request at \url{https://github.com/Cuprate/cuprate/pull/155}, implements:
  \begin{enumerate}
    \item FastSyncService and tests.\\
    \item A tool to generate the hashes of hashes from a synced blockchain.
  \end{enumerate}
\end{frame}

\subsection{Hairy details}
\begin{frame}[fragile]
  \frametitle{Hairy details}
  Hashes of hashes stored as a text file:
  \lstset{
    basicstyle=\tiny\ttfamily,
  }
  \begin{lstlisting}
[
	hex!("1adffbaf832784406018009e07d3dc3a39da7edb6632523c119ed8acb32eb934"),
	hex!("ae960265e3398d04f3cd4f949ed13c2689424887c71c1441a03d900a9d3a777f"),
	hex!("938c72d267bbd3a17cdecbe02443d00012ee62d6e9f3524f5a914192110b1798"),
]
  \end{lstlisting}
  that gets inline included in the source:
  \begin{lstlisting}
static HASHES_OF_HASHES: &[HashOfHashes] = &include!("./data/hashes_of_hashes");
  \end{lstlisting}
  Text file 2x bigger than binary file (checkpoints.dat) in monerod's repo.\\
  Bloat in the final Cuprate binary is the same, code is simpler.
\end{frame}

\subsection{TODO}
\begin{frame}[fragile]
  \frametitle{TODO}
  \begin{itemize}
    \item Full block validation with PoW skipping.
    \item Command line option to enable/disable fast-sync.
    \item Test with full mainnet chain.
    \item Document code.
    \item Get it reviewed and merged.
  \end{itemize}
\end{frame}

\section{Feedback?}
\begin{frame}
  \frametitle{Feedback?}
  \begin{itemize}
    \setlength\itemsep{1em}
    \item Questions
    \item Comments
    \item Ideas
  \end{itemize}
  \vspace{2em}
  All welcomed!
\end{frame}

\begin{frame}
  \frametitle{Licenses}
  {\scriptsize
    \begin{tabular}{l | l | l}
      item & source & license \\
      \hline
      Cuprate logo & \url{https://github.com/Cuprate/cuprate/tree/main/misc/logo} & CC-BY-SA 4.0 \\
    \end{tabular}
  }
\end{frame}

\end{document}
